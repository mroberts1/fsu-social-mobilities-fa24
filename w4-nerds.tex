% Options for packages loaded elsewhere
\PassOptionsToPackage{unicode}{hyperref}
\PassOptionsToPackage{hyphens}{url}
\PassOptionsToPackage{dvipsnames,svgnames,x11names}{xcolor}
%
\documentclass[
  letterpaper,
  DIV=11,
  numbers=noendperiod,
  oneside]{scrartcl}

\usepackage{amsmath,amssymb}
\usepackage{iftex}
\ifPDFTeX
  \usepackage[T1]{fontenc}
  \usepackage[utf8]{inputenc}
  \usepackage{textcomp} % provide euro and other symbols
\else % if luatex or xetex
  \usepackage{unicode-math}
  \defaultfontfeatures{Scale=MatchLowercase}
  \defaultfontfeatures[\rmfamily]{Ligatures=TeX,Scale=1}
\fi
\usepackage{lmodern}
\ifPDFTeX\else  
    % xetex/luatex font selection
\fi
% Use upquote if available, for straight quotes in verbatim environments
\IfFileExists{upquote.sty}{\usepackage{upquote}}{}
\IfFileExists{microtype.sty}{% use microtype if available
  \usepackage[]{microtype}
  \UseMicrotypeSet[protrusion]{basicmath} % disable protrusion for tt fonts
}{}
\makeatletter
\@ifundefined{KOMAClassName}{% if non-KOMA class
  \IfFileExists{parskip.sty}{%
    \usepackage{parskip}
  }{% else
    \setlength{\parindent}{0pt}
    \setlength{\parskip}{6pt plus 2pt minus 1pt}}
}{% if KOMA class
  \KOMAoptions{parskip=half}}
\makeatother
\usepackage{xcolor}
\usepackage[left=1in,marginparwidth=2.0666666666667in,textwidth=4.1333333333333in,marginparsep=0.3in]{geometry}
\setlength{\emergencystretch}{3em} % prevent overfull lines
\setcounter{secnumdepth}{-\maxdimen} % remove section numbering
% Make \paragraph and \subparagraph free-standing
\makeatletter
\ifx\paragraph\undefined\else
  \let\oldparagraph\paragraph
  \renewcommand{\paragraph}{
    \@ifstar
      \xxxParagraphStar
      \xxxParagraphNoStar
  }
  \newcommand{\xxxParagraphStar}[1]{\oldparagraph*{#1}\mbox{}}
  \newcommand{\xxxParagraphNoStar}[1]{\oldparagraph{#1}\mbox{}}
\fi
\ifx\subparagraph\undefined\else
  \let\oldsubparagraph\subparagraph
  \renewcommand{\subparagraph}{
    \@ifstar
      \xxxSubParagraphStar
      \xxxSubParagraphNoStar
  }
  \newcommand{\xxxSubParagraphStar}[1]{\oldsubparagraph*{#1}\mbox{}}
  \newcommand{\xxxSubParagraphNoStar}[1]{\oldsubparagraph{#1}\mbox{}}
\fi
\makeatother


\providecommand{\tightlist}{%
  \setlength{\itemsep}{0pt}\setlength{\parskip}{0pt}}\usepackage{longtable,booktabs,array}
\usepackage{calc} % for calculating minipage widths
% Correct order of tables after \paragraph or \subparagraph
\usepackage{etoolbox}
\makeatletter
\patchcmd\longtable{\par}{\if@noskipsec\mbox{}\fi\par}{}{}
\makeatother
% Allow footnotes in longtable head/foot
\IfFileExists{footnotehyper.sty}{\usepackage{footnotehyper}}{\usepackage{footnote}}
\makesavenoteenv{longtable}
\usepackage{graphicx}
\makeatletter
\def\maxwidth{\ifdim\Gin@nat@width>\linewidth\linewidth\else\Gin@nat@width\fi}
\def\maxheight{\ifdim\Gin@nat@height>\textheight\textheight\else\Gin@nat@height\fi}
\makeatother
% Scale images if necessary, so that they will not overflow the page
% margins by default, and it is still possible to overwrite the defaults
% using explicit options in \includegraphics[width, height, ...]{}
\setkeys{Gin}{width=\maxwidth,height=\maxheight,keepaspectratio}
% Set default figure placement to htbp
\makeatletter
\def\fps@figure{htbp}
\makeatother

% load packages
\usepackage{geometry}
\usepackage{xcolor}
\usepackage{eso-pic}
\usepackage{fancyhdr}
\usepackage{sectsty}
\usepackage{fontspec}
\usepackage{titlesec}

%% Set page size with a wider right margin
\geometry{a4paper, total={170mm,257mm}, left=20mm, top=20mm, bottom=20mm, right=50mm}

%% Let's define some colours
\definecolor{light}{HTML}{E6E6FA}
\definecolor{highlight}{HTML}{800080}
\definecolor{dark}{HTML}{330033}

%% Let's add the border on the right hand side 
\AddToShipoutPicture{% 
    \AtPageLowerLeft{% 
        \put(\LenToUnit{\dimexpr\paperwidth-3cm},0){% 
            \color{light}\rule{3cm}{\LenToUnit\paperheight}%
          }%
     }%
     % logo
    \AtPageLowerLeft{% start the bar at the bottom right of the page
        \put(\LenToUnit{\dimexpr\paperwidth-2.25cm},27.2cm){% move it to the top right
            \color{light}\includegraphics[width=1.5cm]{_extensions/nrennie/PrettyPDF/logo.png}
          }%
     }%
}

%% Style the page number
\fancypagestyle{mystyle}{
  \fancyhf{}
  \renewcommand\headrulewidth{0pt}
  \fancyfoot[R]{\thepage}
  \fancyfootoffset{3.5cm}
}
\setlength{\footskip}{20pt}

%% style the chapter/section fonts
\chapterfont{\color{dark}\fontsize{20}{16.8}\selectfont}
\sectionfont{\color{dark}\fontsize{20}{16.8}\selectfont}
\subsectionfont{\color{dark}\fontsize{14}{16.8}\selectfont}
\titleformat{\subsection}
  {\sffamily\Large\bfseries}{\thesection}{1em}{}[{\titlerule[0.8pt]}]
  
% left align title
\makeatletter
\renewcommand{\maketitle}{\bgroup\setlength{\parindent}{0pt}
\begin{flushleft}
  {\sffamily\huge\textbf{\MakeUppercase{\@title}}} \vspace{0.3cm} \newline
  {\Large {\@subtitle}} \newline
  \@author
\end{flushleft}\egroup
}
\makeatother

%% Use some custom fonts
\setsansfont{Ubuntu}[
    Path=_extensions/nrennie/PrettyPDF/Ubuntu/,
    Scale=0.9,
    Extension = .ttf,
    UprightFont=*-Regular,
    BoldFont=*-Bold,
    ItalicFont=*-Italic,
    ]

\setmainfont{Ubuntu}[
    Path=_extensions/nrennie/PrettyPDF/Ubuntu/,
    Scale=0.9,
    Extension = .ttf,
    UprightFont=*-Regular,
    BoldFont=*-Bold,
    ItalicFont=*-Italic,
    ]
\KOMAoption{captions}{tableheading}
\makeatletter
\@ifpackageloaded{caption}{}{\usepackage{caption}}
\AtBeginDocument{%
\ifdefined\contentsname
  \renewcommand*\contentsname{Table of contents}
\else
  \newcommand\contentsname{Table of contents}
\fi
\ifdefined\listfigurename
  \renewcommand*\listfigurename{List of Figures}
\else
  \newcommand\listfigurename{List of Figures}
\fi
\ifdefined\listtablename
  \renewcommand*\listtablename{List of Tables}
\else
  \newcommand\listtablename{List of Tables}
\fi
\ifdefined\figurename
  \renewcommand*\figurename{Figure}
\else
  \newcommand\figurename{Figure}
\fi
\ifdefined\tablename
  \renewcommand*\tablename{Table}
\else
  \newcommand\tablename{Table}
\fi
}
\@ifpackageloaded{float}{}{\usepackage{float}}
\floatstyle{ruled}
\@ifundefined{c@chapter}{\newfloat{codelisting}{h}{lop}}{\newfloat{codelisting}{h}{lop}[chapter]}
\floatname{codelisting}{Listing}
\newcommand*\listoflistings{\listof{codelisting}{List of Listings}}
\makeatother
\makeatletter
\makeatother
\makeatletter
\@ifpackageloaded{caption}{}{\usepackage{caption}}
\@ifpackageloaded{subcaption}{}{\usepackage{subcaption}}
\makeatother
\makeatletter
\@ifpackageloaded{tcolorbox}{}{\usepackage[skins,breakable]{tcolorbox}}
\makeatother
\makeatletter
\@ifundefined{shadecolor}{\definecolor{shadecolor}{rgb}{.97, .97, .97}}{}
\makeatother
\makeatletter
\@ifundefined{codebgcolor}{\definecolor{codebgcolor}{named}{light}}{}
\makeatother
\makeatletter
\ifdefined\Shaded\renewenvironment{Shaded}{\begin{tcolorbox}[enhanced, frame hidden, colback={codebgcolor}, boxrule=0pt, breakable, sharp corners]}{\end{tcolorbox}}\fi
\makeatother
\makeatletter
\@ifpackageloaded{sidenotes}{}{\usepackage{sidenotes}}
\@ifpackageloaded{marginnote}{}{\usepackage{marginnote}}
\makeatother

\ifLuaTeX
  \usepackage{selnolig}  % disable illegal ligatures
\fi
\usepackage{bookmark}

\IfFileExists{xurl.sty}{\usepackage{xurl}}{} % add URL line breaks if available
\urlstyle{same} % disable monospaced font for URLs
\hypersetup{
  colorlinks=true,
  linkcolor={highlight},
  filecolor={Maroon},
  citecolor={Blue},
  urlcolor={highlight},
  pdfcreator={LaTeX via pandoc}}


\author{}
\date{}

\begin{document}

\pagestyle{mystyle}


\href{https://www.wired.com/2015/06/no-matter-reddit-going-alienate-people/}{\includegraphics{img/reddit-alien.webp}}

\subsection{W4: Brotopia: Y Combinator, Reddit, and the Rise of Nerd
Culture}\label{w4-brotopia-y-combinator-reddit-and-the-rise-of-nerd-culture}

\begin{quote}
``Make something people want.''

---Paul Graham
\end{quote}

\begin{quote}
And you may ask yourself\\
``Well, how did I get here?''

---Talking Heads, ``Once in a Lifetime''
\end{quote}

Continuing our exploration of the origins of today's social mediascape,
this week we look at the rise of \href{https://www.reddit.com/}{Reddit},
in Christine Lagorio-Chafkin's fascinating history of the platform,
\emph{We Are The Nerds}. Part of the purpose of the readings we've been
doing for the past few weeks is very simple: to understand how we got
here, in other words, the process by which the social mediascape that we
are all so familiar with came to be. As we see both in Parmy Olson's
book about ChatGPT and \emph{We Are The Nerds}, the countours of that
world began to be drawn in the mid-2000s, when the social mediascape was
still taking shape. One of the most important sites in this process was
Reddit, which was launched in 2005 and quickly became a hub for online
communities and discussions.

One of the most striking things about the origins of social media
startups that emerges from these books that we're reading is how
male-dominated they are. In that respect, it's significant that both the
book about ChatGPT that we were reading last week and this week's book
about Reddit are authored by women, who cast a sardonic eye on the
male-dominated world of geeky or nerdy programmers, from Bill Gates to
Mark Zuckerberg to Sam Altman, who have played such a role in shaping
the social mediascape that most of us now live in. Like it or not, are
all nerds now. Today, arguably, the tech world is more diverse than it
has ever been, compared to several decades ago--prominent female coders
include \href{https://mollywhite.net/}{Molly White},
\href{https://maggieappleton.com/}{Maggie Appleton}, or French UX
designer \href{https://stephaniewalter.com/}{Stephanie Walter}. There
are movements in many cities around the globe that offer coding
workshops specifically for women, such as organizations like
\href{https://girlswhocode.com/}{\textbf{Girls Who Code}}.

Social media startups, why they mostly fail, but sometimes achieve
astonishing success, is of course something that everyone has been
trying to figure out for decades now, and even though the process is
understood much better today, there is still no magic bullet. In the
early 2000s, even before social media platforms as we know them existed,
I remember watching a documentary film called \emph{Startup.com} (2001),
which followed the failure of a new-media startup at the time of the
bursting of the dot.com bubble around the turn of the millennium. Since
then, in recent years there has been an explosion in the historicization
of the rise of social media startups, and often their equally
precipitous fall. At this point, it's become a new genre of
\textbf{social media entertainment}, including Netflix/Apple/Max shows
and documentaries like:

\begin{itemize}
\tightlist
\item
  \href{https://youtu.be/_pXyTCYdzBA}{\textbf{Living The Stream}}
  (2019), a documentary about Twitch;
\item
  \href{https://youtu.be/VMP21LO0Guc}{\textbf{Super Pumped: The Battle
  for Uber}} (2022), on Netflix;
\item
  \href{https://youtu.be/UREIAoL0Spk}{\textbf{WeCrashed}} (2022), about
  the rise and fall of the office-sharing company WeWork, on Apple TV;
\item
  \href{https://youtu.be/utDDwu6zfTc}{\textbf{The Thinking Game}}
  (2023), about Demis Hassabis's AI company DeepMind.
\end{itemize}

I encourage you to check out some of these shows and/or documentaries,
and you would be welcome to include any of them in the remaining
assignments for the course.

One of the most surprising things that emerges from the readings from
last week and this is the influential role of
\href{https://www.ycombinator.com/}{Y Combinator}, Paul Graham's startup
incubator, whose alums include--as we saw last week--both Sam Altman,
the creator of ChatGPT, and Alexis Ohanian and Steve Huffman, the
founders of Reddit. Take a look at the website and the list of companies
that have emerged from Y Combinator, and it's hard not to be impressed
by the sheer number of successful companies that have been launched
through the program. Moreover, it's still continuing---perhaps an
interesting opening question for you might be this: if you were going to
start a new social media platform today, what would it be, and how would
you pitch it to Y Combinator?

If nothing else, \emph{We Are The Nerds} provides a useful introduction
to the language of new-media startups, most notably Paul Graham's
concept of the \textbf{minimum viable product} (MVP) (p.~50), the
practice of launching an online product as early as possible in order to
test the market and gather feedback from users. As you will recall, Sam
Altman's approach to AI with ChatGPT has followed exactly the same model
(in contrast to Demis Hassabis's with DeepMind), although arguably given
its nature the risks are much higher than for the simple launch of a
social media news platforms. I wonder how far you think Graham's MVP
approach might apply to AI---was Altman irresponsible for launching such
a powerful and potentially dangerous product on the internet before it
had been propertly tested?

Another useful term that may be new to you is \textbf{growth hacking}
(p.~54), in reference to Ohanian and Huffman's strategy of creating fake
accounts that they used in the early days of Reddit to give the
impression of early user activity on the platform and prime the pump of
subscriptions.

I was wondering how many people in the class (if any) are familiar with
some of the older social bookmarking sites that are mentioned in the
book, notably
\href{https://en.wikipedia.org/wiki/Delicious_(website)}{Delicious} and
\href{https://en.wikipedia.org/wiki/Digg}{Digg}? Do any of you use these
platforms today? Delicious (real name del.icio.us) was one of the first
social bookmarking sites that I used, but I used it entirely simply to
save my own bookmarks rather than to share them with others. The
platform was acquired by Yahoo in 2005 for \$20 million. Sites like
Delicious may seem old and clunky today, but they remain useful: one of
my favorites is \href{https://pinboard.in/}{\textbf{Pinboard}}; founded
by the software developer and internet activist Maciej Cegłowski, it's a
simple, open-source, and privacy-respecting alternative to Delicious.

Social bookmarking sites today are most often likely to be web apps. An
example is \href{https://webmarks.5apps.com/about}{\textbf{Webmarks}},
which I tend to use to store important links that I want to remember.
Webmarks is part of a new generation of open-source apps called
\href{https://0data.app/glance}{\textbf{zero data apps}}, an initiative
intended to return control of our data to the user. I've tried quite a
few of the apps on this page, and still use a number of them. Your
private data is typically either hosted by an external service such as
\href{https://5apps.com/about}{\textbf{5apps}}, or is self-hosted. I
have tried a number of these apps and like them, especially
\href{https://hyperdraft.app/}{\textbf{Hyperdraft}} and
\href{https://noeldemartin.github.io/media-kraken/login}{\textbf{Media
Kraken}}, a zero-date alternative to social movie platforms like
\href{https://letterboxd.com/}{Letterboxd}. Setup can be a little tricky
because many of them are decentralized peer-to-peer apps, but if you
like the idea of getting back control over your own data I recommend
them.

So my last question is also the most obvious: do you use Reddit yourself
today? If so, what do you use it for?

\begin{center}\rule{0.5\linewidth}{0.5pt}\end{center}




\end{document}
